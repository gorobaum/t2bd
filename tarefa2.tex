\documentclass[conference]{IEEEtran}

\usepackage{array}
\usepackage[brazil]{babel}
\usepackage[T1]{fontenc}
\usepackage[utf8]{inputenc}
\hyphenation{op-tical net-works semi-conduc-tor}


\begin{document}

\title{Análise prática sobre estado atual dos bancos de dados ativos}

\author{\IEEEauthorblockN{Caio de freitas Valente, Gabriel Reganati,  Rafael Reggiani Manzo, Thiago de Gouveia Nunes}
\IEEEauthorblockA{Instituto de Matemática e Estatística\\
Universidade de São Paulo\\
São Paulo, São Paulo\\
Email: thiago.gouveia.nunes@gmail.com}
}

\maketitle
\IEEEpeerreviewmaketitle

\section{Introdução}
  \subsection{Motivação e contextualização}
	Passar nessa porra.

  \subsection{Conceitos}

\section{Estado da arte}
  \subsection{Trabalhos científicos}

  \subsection{Ferramentas}

\section{Estudo de caso}
  \subsection{Características técnicas}
    \subsubsection{SQLServer}
    \subsubsection{MySQL}

  \subsection{Cenário}

  \subsection{Modelagens}
    \subsubsection{SQLServer}
    \subsubsection{MySQL}

  \subsection{Descição dos testes}

  \subsection{Resultados}
    \subsubsection{SQLServer}
    \subsubsection{MySQL}


\begin{thebibliography}{1}

\bibitem{IEEEhowto:kopka}
H.~Kopka and P.~W. Daly, \emph{A Guide to \LaTeX}, 3rd~ed.\hskip 1em plus
  0.5em minus 0.4em\relax Harlow, England: Addison-Wesley, 1999.

\end{thebibliography}

\end{document}

